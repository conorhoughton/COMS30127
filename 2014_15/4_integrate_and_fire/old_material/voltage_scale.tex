\documentclass[11pt,a4paper]{scrartcl}
\typearea{12}
\usepackage{graphicx}
\usepackage{pstricks}
\usepackage{listings}
\lstset{language=python}
\pagestyle{headings}
\markright{Computation Neuroscience - voltage scale}
\begin{document}

There is an interesting argument given in Dayan and Abbott's book that
explains the voltage scales for the electrodynamics of neurons; it is
useful because it touches on themes we will return to in our study of
neurons. Basically we will see that neurons work partly due to
diffusion, which in turn depends on ions fly around because of their
thermal energy. We know the thermal energy of an ion at temperature
$T$, it is $k_BT$ where $k_B$ is the Boltzmann constant. Now, consider
the potential difference with that corresponding energy, the energy
required to move an ion of charge one across a voltage $V_0$ is
$qV_0$, so for neurons to work we would expect the scale of the
voltages involve to be of the order where the thermal energy was
similar to the energy required to overcome the voltage gap, that is,
we expect the voltage gap to be able to modulate that flow. Hence
$qV_0\approx k_BT$ or
\begin{equation}
V_0\approx \frac{k_BT}{q}\approx 27\,\mbox{mV}
\end{equation}
at room temperature.

\end{document}

