\documentclass[11pt,a4paper]{scrartcl}
\typearea{12}
\usepackage{graphicx}
\usepackage{pstricks}
\usepackage{listings}
\lstset{language=python}
\pagestyle{headings}
\markright{Computation Neuroscience - Lecture 4}
\begin{document}

\subsection*{Introduction}

The notes contain a figure, repeated here as Fig.~\ref{f_i_curve}
showing the f-I curve for an integrate and fire neuron. To save time
the shape of the curve was never derived but for the curious I have
writen it out here.

In the model 
\begin{equation}
\tau_m\frac{dV}{dt}=E_L-V+R_mI_e
\end{equation}
which we can solve from our study of odes, it gives
\begin{equation}
V(t)=E_L+R_mI_e+[V(0)-E_L-R_mI_e]e^{-t/\tau_m}
\end{equation}
so if the neuron has spiked and is reset at time $t=0$ and reaches
threshold at time $t=T$, assume $V_R=E_L$ we have
\begin{equation}
V_T=E_L+R_mI_e-R_mI_ee^{-T/\tau_m}
\end{equation}
so 
\begin{equation}
e^{-T/\tau_m}=\frac{E_L+R_mI_e-V_T}{R_mI_e}
\end{equation}
Taking the log of both sides we get
\begin{equation}
T=\tau_m\log\left[\frac{R_mI_e}{E_L+R_mI_e-V_T}\right]
\end{equation}
Finally, this is the time between spikes, so the frequency is one over this. It is only defined for $R_mI_e>V_T-E_L$, below that there is no spiking and the frequency is zero. The actual gnuplot command used to make the figure was\\
\texttt{plot [0:22] x>15 ? 1/(.01*log(x/(x-15))) : 0}

\begin{figure}
\begin{center}
% GNUPLOT: LaTeX picture with Postscript
\begingroup
  \makeatletter
  \providecommand\color[2][]{%
    \GenericError{(gnuplot) \space\space\space\@spaces}{%
      Package color not loaded in conjunction with
      terminal option `colourtext'%
    }{See the gnuplot documentation for explanation.%
    }{Either use 'blacktext' in gnuplot or load the package
      color.sty in LaTeX.}%
    \renewcommand\color[2][]{}%
  }%
  \providecommand\includegraphics[2][]{%
    \GenericError{(gnuplot) \space\space\space\@spaces}{%
      Package graphicx or graphics not loaded%
    }{See the gnuplot documentation for explanation.%
    }{The gnuplot epslatex terminal needs graphicx.sty or graphics.sty.}%
    \renewcommand\includegraphics[2][]{}%
  }%
  \providecommand\rotatebox[2]{#2}%
  \@ifundefined{ifGPcolor}{%
    \newif\ifGPcolor
    \GPcolorfalse
  }{}%
  \@ifundefined{ifGPblacktext}{%
    \newif\ifGPblacktext
    \GPblacktexttrue
  }{}%
  % define a \g@addto@macro without @ in the name:
  \let\gplgaddtomacro\g@addto@macro
  % define empty templates for all commands taking text:
  \gdef\gplbacktext{}%
  \gdef\gplfronttext{}%
  \makeatother
  \ifGPblacktext
    % no textcolor at all
    \def\colorrgb#1{}%
    \def\colorgray#1{}%
  \else
    % gray or color?
    \ifGPcolor
      \def\colorrgb#1{\color[rgb]{#1}}%
      \def\colorgray#1{\color[gray]{#1}}%
      \expandafter\def\csname LTw\endcsname{\color{white}}%
      \expandafter\def\csname LTb\endcsname{\color{black}}%
      \expandafter\def\csname LTa\endcsname{\color{black}}%
      \expandafter\def\csname LT0\endcsname{\color[rgb]{1,0,0}}%
      \expandafter\def\csname LT1\endcsname{\color[rgb]{0,1,0}}%
      \expandafter\def\csname LT2\endcsname{\color[rgb]{0,0,1}}%
      \expandafter\def\csname LT3\endcsname{\color[rgb]{1,0,1}}%
      \expandafter\def\csname LT4\endcsname{\color[rgb]{0,1,1}}%
      \expandafter\def\csname LT5\endcsname{\color[rgb]{1,1,0}}%
      \expandafter\def\csname LT6\endcsname{\color[rgb]{0,0,0}}%
      \expandafter\def\csname LT7\endcsname{\color[rgb]{1,0.3,0}}%
      \expandafter\def\csname LT8\endcsname{\color[rgb]{0.5,0.5,0.5}}%
    \else
      % gray
      \def\colorrgb#1{\color{black}}%
      \def\colorgray#1{\color[gray]{#1}}%
      \expandafter\def\csname LTw\endcsname{\color{white}}%
      \expandafter\def\csname LTb\endcsname{\color{black}}%
      \expandafter\def\csname LTa\endcsname{\color{black}}%
      \expandafter\def\csname LT0\endcsname{\color{black}}%
      \expandafter\def\csname LT1\endcsname{\color{black}}%
      \expandafter\def\csname LT2\endcsname{\color{black}}%
      \expandafter\def\csname LT3\endcsname{\color{black}}%
      \expandafter\def\csname LT4\endcsname{\color{black}}%
      \expandafter\def\csname LT5\endcsname{\color{black}}%
      \expandafter\def\csname LT6\endcsname{\color{black}}%
      \expandafter\def\csname LT7\endcsname{\color{black}}%
      \expandafter\def\csname LT8\endcsname{\color{black}}%
    \fi
  \fi
  \setlength{\unitlength}{0.0500bp}%
  \begin{picture}(5040.00,3528.00)%
    \gplgaddtomacro\gplbacktext{%
      \csname LTb\endcsname%
      \put(814,704){\makebox(0,0)[r]{\strut{} 0}}%
      \put(814,1024){\makebox(0,0)[r]{\strut{} 10}}%
      \put(814,1344){\makebox(0,0)[r]{\strut{} 20}}%
      \put(814,1664){\makebox(0,0)[r]{\strut{} 30}}%
      \put(814,1984){\makebox(0,0)[r]{\strut{} 40}}%
      \put(814,2303){\makebox(0,0)[r]{\strut{} 50}}%
      \put(814,2623){\makebox(0,0)[r]{\strut{} 60}}%
      \put(814,2943){\makebox(0,0)[r]{\strut{} 70}}%
      \put(814,3263){\makebox(0,0)[r]{\strut{} 80}}%
      \put(946,484){\makebox(0,0){\strut{} 0}}%
      \put(1786,484){\makebox(0,0){\strut{} 5}}%
      \put(2626,484){\makebox(0,0){\strut{} 10}}%
      \put(3467,484){\makebox(0,0){\strut{} 15}}%
      \put(4307,484){\makebox(0,0){\strut{} 20}}%
      \put(176,1983){\rotatebox{-270}{\makebox(0,0){\strut{}firing rate in Hz}}}%
      \put(2794,154){\makebox(0,0){\strut{}$R_mI_e$}}%
    }%
    \gplgaddtomacro\gplfronttext{%
    }%
    \gplbacktext
    \put(0,0){\includegraphics{f_i_curve}}%
    \gplfronttext
  \end{picture}%
\endgroup

\end{center}
\caption{The firing rate, that is spikes per second, for the integrate
  and fire neuron with different constant inputs with $\tau_m=10$ ms,
  $V_T=-55$ mV and both the leak and reset given by $-70$ mV. Notice
  how there is no firing until a threshold is reached and after that
  the firing increases very quickly. \label{f_i_curve}}
\end{figure}

\end{document}

