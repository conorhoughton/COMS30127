\documentclass[11pt,a4paper]{scrartcl}
\typearea{12}
\usepackage{graphicx}
\usepackage{pstricks}
\usepackage{listings}
\usepackage{amsmath}
\usepackage{tikz}
%\usepackage{tikzscale}
%\usepackage{pgfplots}

%\pgfplotsset{compat=1.8}


\usepackage{fancyhdr}
\pagestyle{fancy}
\lfoot{\texttt{github.com/conorhoughton/COMS30127}}
\lhead{Computation Neuroscience - 14\_phase\_analysis (j/k) - Conor}
\rhead{\thepage}
\cfoot{}

\begin{document}

\section*{Phase analysis}
To recapt, we have seen the Hodgkin Huxley equation
\begin{equation}
C_m\frac{dV}{dt}=g_{\text{\l}}(E_{\text{\l}}-V)+g_{\text{Na}}(E_{\text{Na}}-V)+g_{\text{K}}(E_{\text{K}}-V)+I
\end{equation}
where the sodium and potassium conductances, $g_{\text{Na}}$ and
$g_{\text{K}}$ have complicated non-linear dynamics. In fact, the
Hodgkin-Huxley equation is really four equation, an equation for $V$
along with equations for the three gating variables $n$, $m$ and $h$, each of the form
\begin{equation}
\frac{dk}{dt}=\alpha_k(1-k)-\beta_k k
\end{equation}
with $k$ standing in for $n$, $m$ or $h$ and $\alpha_k(V)$ and $\beta_k(V)$
being the probability of going from closed to open and open to
closed. By moving stuff around this can be easily rewritten in a familar form
\begin{equation}
\tau_k\frac{dk}{dt}=k_\infty - k
\end{equation}
with 
\begin{equation}
\tau_k=\frac{1}{\alpha_k+\beta_k}
\end{equation}
and
\begin{equation}
k_\infty=\frac{\alpha}{\alpha_k+\beta_k}
\end{equation}
Hence the ion channels relax towards some asymptotic value $n_\infty$,
$m_\infty$ and $h_\infty$ with some time scale $\tau_n$, $\tau_m$ and
$\tau_h$; however all these quantities depend on the voltage so the
equations are coupled to the voltage equation.

When it is recognized as a system of four non-linear differential
equation it is clear that it may prove hard to analyze the
Hodgkin-Huxley equation; for example, the phase space is
four-dimensional for a start, making it hard to picture. For this
reason it is common to simplify the Hodgkin-Huxley equation in the
hope of getting some insight into its behaviour, this is important,
for example, if you are interested in getting an intuitive
understanding of how different neuronal models can support the
different behaviours of observed in neurons: some neurons spike
continuously, some don't; some burst, that is, switch back and forth
between high spiking and low spiking statees.

The goal then is look at models that approximate the Hodgkin-Huxley
equation and simplify while keeping it complex enough so that it is
still a rich enough to model spiking.

\subsection*{The Morris-Lacar model}

The key idea behind the Morris-Lecar model \cite{MorrisLecar1981} is that $\tau_m$ is very
small. When we looked at the behaviour of equations like the equation
for $m$ we saw that the functions track their asymptotic value, with
the $\tau$ value governing how closely it succeeds in reaching the
equilibium situation where $m$ equals $m_\infty$. Thus in the
Morris-Lecar model $m^3$ is replaced by an asmyptotic value. Next the
effect of $h$ is ignored, or lumped in with $n$. Altogether this gives
a two-dimensional model of the neuron which is much easier to think
about.  The model is simplified further by ignoring the indicies on
the gating variables, so the single gating variable appears with a
single power.

Just as the Hodgkin-Huxley model is a model of a specific axon, the
squid giant axon, that is adapted to wider use, the Morris-Lecar is a
model of a muscle fibre in the barnacle. Like Hodgkin and Huxley, they
wrote down an equation of the form they expected to work and then
adjust parameters to fit the actual data. In the barnacle the main ion
responsible for depolarization is calcium rather than by sodium, so
the model has calcium rather than sodium, in applying the model to
other neurons this could be changed.

The Morris-Lacer model is
\begin{equation}
C_m\frac{dV}{dt}=g_{\text{l}} (E_{\text{l}}-V)+g_{\text{m}} 




\begin{thebibliography}{99}
\bibitem{MorrisLecar1981}
\newblock Morris, C and Lecar, H (1981),
\newblock Voltage Oscillations in the barnacle giant muscle fiber
\newblock Biophys. J., 35:93--213,
\end{thebibliography}

\end{document}

