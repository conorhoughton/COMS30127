\documentclass[12pt]{article}
\usepackage{amsfonts, epsfig}

\usepackage{graphicx}
\usepackage{fancyhdr}
\pagestyle{fancy}
\lfoot{\texttt{github.com/conorhoughton/COMS30127}}
\lhead{Computation Neuroscience - Coursework 2 - spike trains}
\rhead{\thepage}
\cfoot{}
\begin{document}

\section*{Coursework 2}

This coursework relates to the properties of spike train. These
questions areadapted from the exercises given in Part 1.1 of
\texttt{http://www.gatsby.ucl.ac.uk/~dayan/book/exercises.html}; these
are exercises for the book Dayan and Abbott, a recommended text book
for this course.


\subsection*{Question 1}

In the \texttt{spike\_trains} folder you will find programmes
\texttt{poisson.jl} and \texttt{poisson.py}. These contain a function
which will generate spike trains simulated using a Poisson process
with a refractory period. Calculate the Fano factor and coefficient of
variation for 1000 seconds of spike train with a firing rate of 35 Hz,
both with no refractory period and with a refractory period of 5 ms.

\subsection*{Question 2}

In the \texttt{spike\_trains} folder you will find the data file
\texttt{rho.dat}. This is contains data collected and provided by Rob
de Ruyter van Steveninck from a fly H1 neuron responding to an
approximate white-noise visual motion stimulus. Data were collected
for 20 minutes at a sampling rate of 500 Hz. In the file, rho is a
vector that gives the sequence of spiking events or nonevents at the
sampled time, that is, every 2 ms. Calculate the Fano factor and coefficient of variation for this spike train.

\subsection*{Question 3} 

In the \texttt{spike\_trains} folder you will find the data file
\texttt{stim.dat}. This give the motion stimulus that evoked the spike
train in \texttt{rho.dat}. Calculate the spike triggered average.

\subsection*{COMSM20127}

Calculate the stimulus triggered by pairs of spikes.

\subsection*{Submission instructions}

\end{document}
